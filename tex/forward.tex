\chapter{序}

看了本书的样稿后使人感到印象深刻。本书充分反映了\TeX{} 的最新进展,尽管\TeX{} 的生命力是顽强的,\TeX{}  的基本命令系统也是稳定的,但是它对非西方语言的扩展以及输出格式等都随着计算机技术的发展以及科技文献传播方式的变化而不断推陈出新,这也正是\TeX{} 能经久不衰的生命力所在。因此推广\TeX{}的书也需要与时俱进。我们写的《\LaTeX 入门与提高》的第二版至今已有7年了,可惜它的作者或者已退休,或者兴趣转移,不可能再作更新。我一直期待能有人出来写一本反映最新发展的\TeX 入门书作为我们那本书的补充及更新。现在看到了刘海洋的《\LaTeX{} 入门》,觉得这正是我所期望的,甚至超过了我的期望。本书文笔活泼,阅读起来像是面对一位向你细细讲解的和蔼老师,他了解你的需求和会遇到的难点,使你爱不择手,而不像一般的软件说明书,只管板着脸罗列一大堆用法,不管你是否需要或是否理解。但是本书作者又很严谨,许多内容都有出处,好像一篇科研论文。不过说到底,这是一本面向读者需求的学习指导书,并非\TeX 的说明书。这正是想学习\TeX 的人最想要的书。而且第8章还讲到了更深入的技巧。因此本书的适用范围可以从初学者直至想自己设计版面或宏的高级应用者。大家都能从本书学到很多东西。尽管国内在\TeX 的普及与发展方面与西方发达国家相比还有很大的差距,但是感谢许多热心的\TeX 爱好者及他们的网站的努力,\TeX 在中国的推广也是富有成效的。越来越多的研究生用\TeX 写作论文或向期刊投稿,并且在答辩或演示时也广泛使用\TeX 生成的PDF。希望本书的出版能为\TeX 在中国的普及作出新的贡献。

\bigskip

\begin{flushright}
    \parbox{\widthof{华东师范大学数学系教授}}{%
    陈志杰\\
    华东师范大学数学系教授\\
    \raisebox{-1ex}{2013年3月5日}}
\end{flushright}